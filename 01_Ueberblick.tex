\documentclass{beamer}
	
\beamertemplatenavigationsymbolsempty

\usepackage[utf8]{inputenc}
\usepackage[german]{babel}
\usepackage{hyperref}
\usepackage{color, colortbl}
\usepackage[noend]{algpseudocode}

\usepackage{beamercmds}
\usepackage{listingcmds}

\AtBeginLecture{}

\AtBeginPart{
  \partcontentframe
}

\AtBeginSection{
  \sectioncontentframe
}

%\AtBeginSubsection{
%  \subsectioncontentframe
%}

\newcommand{\positive}{\color{ForestGreen}\textbf{+}}
\newcommand{\negative}{\color{Red}\textbf{--\,}}
\renewcommand{\gets}{:=}

\usepackage{tikz}
\usepackage{tikz-uml}
\usetikzlibrary{positioning}

\author{Reiner Hüchting}
\title{Informatik 2}

\subtitle{Themenüberblick}

\begin{document}

\maketitle

\begin{frame}
\frametitle{Themenüberblick}
\begin{block}{Algorithmen}\end{block}
\begin{block}{Datenstrukturen}\end{block}
\begin{block}{Entwurfstechniken}\end{block}
\begin{block}{Komplexitätsabschätzungen}\end{block}
\end{frame}

\part{Algorithmen}

\section{Suchverfahren}
\subsection{lineare Suche}
\subsection{binäre Suche}
\subsection{Komplexität}

\section{Sortierverfahren}
\subsection{einfache Sortierverfahren}
\subsection{Mergesort, Quicksort}
\subsection{nicht vergleichsbasiertes Sortieren}
\subsection{Komplexität}

\part{Datenstrukturen}

\section{Arrays und Listen}
\subsection{Klassische Arrays: Vor- und Nachteile}
\subsection{Dynamische Arrays}
\subsection{Verkettete Listen}
\subsection{Speziellere Listen-Datentypen}
\subsection{Komplexität}

\section{Baumstrukturen}
\subsection{binäre Suchbäume, AVL, Red-Black-Trees}
\subsection{Heaps, Heapsort}
\subsection{Präfixbäume}
\subsection{Komplexität}

\section{Graphen}
\subsection{Definition, Anwendungen}
\subsection{typische Graphalgorithmen}

\part{Entwurfstechniken und Komplexitätsabschätzungen}

\section{Objektorientierung, Komposition, Vererbung}
\subsection{"nebenbei" durch Übungsaufgaben/Codebeispiele}
\subsection{ggf. Beispiele aus anderen Programmiersprachen}

\section{Komplexitätsabschätzungen}
\subsection{Notation und Analyse der Komplexität von Algorithmen}

\begin{frame}
\end{frame}

\end{document}