\begin{frame}
\frametitle{\insertsection}
\begin{definition}[AVL-Baum]
	Ein \alert{AVL-Baum} ist ein binärer Suchbaum, bei dem der Balancefaktor jedes Knotens im Bereich
	$\{-1,0,1\}$ liegt.
\end{definition}

\begin{block}<2->
{Erhaltung der AVL-Eigenschaft}
\begin{itemize}
	\item Beim Einfügen oder Löschen kann die Eigenschaft verloren gehen.
	\item Der Baum (oder ein Teilbaum) muss \alert{rotiert} werden.
	\item Intuitiv: \alert{nach rechts Rotieren} bedeutet, die Wurzel in den rechten Teilbaum zu verschieben
			und eine neue Wurzel aus dem linken Teilbaum zu holen.
\end{itemize}
\end{block}
\end{frame}

\begin{frame}
\frametitle{\insertsection}
\begin{block}
{Rotationsarten (\alert{Einfachrotationen})}
\begin{itemize}
	\item Links-Rotation:
	\begin{itemize}
		\item Wurzel wird in den linken Teilbaum abgesenkt.
		\item Rechtes Kind wird die neue Wurzel.
		\item Linkes Kind der neuen Wurzel wird zum rechten Kind der alten.
	\end{itemize}
	\item<2-> Rechts-Rotation:
	\begin{itemize}
		\item Wurzel wird in den rechten Teilbaum abgesenkt.
		\item Linkes Kind wird die neue Wurzel.
		\item Rechtes Kind der neuen Wurzel wird zum linken Kind der alten.
	\end{itemize}
\end{itemize}
\onslide<3->{Diese beiden Rotationen stellen die Balance wieder her, wenn das Ungleichgewicht \alert{ganz außen} im Baum ist.}
\end{block}
\end{frame}

\begin{frame}
\frametitle{\insertsection}
\begin{block}
{Ungleichgewichtssituationen}
Wir unterscheiden, auf welcher Seite des Baumes das Ungleichgewicht besteht:
\end{block}

\begin{block}
{Links-Links}
\begin{itemize}
	\item Balancefaktoren der Wurzel und des linken Kindes negativ.
	\item Balance wird durch Rechtsrotation wieder hergerstellt.
\end{itemize}
\end{block}
\begin{block}<2->
{Rechts-Rechts}
\begin{itemize}
	\item Balancefaktoren der Wurzel und des rechten Kindes positiv.
	\item Balance wird durch Linkssrotation wieder hergerstellt.
\end{itemize}
\end{block}
\end{frame}

\begin{frame}
\frametitle{\insertsection}
\begin{block}
{Ungleichgewichtssituationen}
Entsprechend gibt es noch die Situationen \emph{Links-Rechts} und \emph{Rechts-Links}:
\end{block}

\begin{block}
{Links-Rechts}
\begin{itemize}
	\item Balancefaktor der Wurzel negativ.
	\item Balancefaktor des linken Kindes positiv.
	\item Balance wird durch \alert{Links-Rechts-Rotation} wieder hergerstellt:
	\begin{enumerate}
		\item Linksrotation durch das linke Kind.
		\item Rechtsrotation durch die Wurzel.
	\end{enumerate}
\end{itemize}
\end{block}
\end{frame}

\begin{frame}
\frametitle{\insertsection}
\begin{block}
{Ungleichgewichtssituationen}
Entsprechend gibt es noch die Situationen \emph{Links-Rechts} und \emph{Rechts-Links}:
\end{block}

\begin{block}
{Rechts-Links}
\begin{itemize}
	\item Balancefaktor der Wurzel positiv.
	\item Balancefaktor des rechten Kindes negativ.
	\item Balance wird durch \alert{Rechts-Links-Rotation} wieder hergerstellt:
	\begin{enumerate}
		\item Rechtsrotation durch das rechte Kind.
		\item Linksrotation durch die Wurzel.
	\end{enumerate}
\end{itemize}
\end{block}
\end{frame}

\begin{frame}
\frametitle{\insertsection}
\begin{block}
{Implementierung von AVL-Bäumen}
\begin{itemize}
	\item Einfügen und Löschen wie bisher.
	\item Dabei zusätzlich Balancefaktoren berechnen.
	\item<3-> Sobald ein Knoten mit Balancefaktor $-2$ gefunden wird, linkes Kind prüfen:
	\begin{itemize}
		\item Kind hat Balancefaktor $-1$: Rechtsrotation
		\item Kind hat Balancefaktor $+1$: Links-Rechts-Rotation
	\end{itemize}
	\item<4-> Sobald ein Knoten mit Balancefaktor $+2$ gefunden wird, rechtes Kind prüfen:
	\begin{itemize}
		\item Kind hat Balancefaktor $+1$: Linksrotation
		\item Kind hat Balancefaktor $-1$: Rechts-Links-Rotation
	\end{itemize}
\end{itemize}
\end{block}
\end{frame}