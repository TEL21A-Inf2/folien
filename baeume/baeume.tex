

\begin{frame}
\frametitle{\insertsection}
\begin{block}
{Wiederholung: Binäre Suche}
\begin{itemize}
	\item[\positive] Halbierung des Suchraums in jedem Schritt
	\item[\negative] Liste muss sortiert sein
	\item[\negative] Nachträgliches Sortieren ist keine Option (\alert{zu langsam})
\end{itemize}
\end{block}
\begin{block}<2->
{Idee: Elemente direkt an der richtigen Stelle einfügen.}
\begin{itemize}
	\item Bei Arrays zwei Möglichkeiten:
		\begin{enumerate}
			\item Richtige Stelle suchen und dann Elemente verschieben.
			\item Vertauschungen wie z.B.\ bei Insertionsort.
		\end{enumerate}
	\item Bei verketteten Listen kann direkt eingefügt werden. 
\end{itemize}
\end{block}
\end{frame}

\begin{frame}
\frametitle{\insertsection}
\begin{block}
{Gesucht: Datenstruktur für effizientes Einfügen von Elementen}
\begin{itemize}
	\item Kein Verschieben von Elementen
	\item Suchraum sollte mit jedem Schritt halbiert werden.
\end{itemize}
\end{block}
\begin{block}<2->
{Idee:}
\begin{itemize}
	\item Pointerstruktur wie bei verketteten Listen.
	\item Jedes Element hat zwei Nachfolger:
		\begin{enumerate}
			\item kleinere Elemente
			\item größere Elemente
		\end{enumerate}
	\item i.W.\ immer noch eine verkettete Liste
		\begin{itemize}
			\item Struktur reflektiert das Verhalten der Suche.
		\end{itemize}
\end{itemize}
\end{block}
\end{frame}

\begin{frame}
\frametitle{\insertsection}
\begin{definition}[Graph]
Ein \alert{Graph} ist ein Tupel $(V,E)$ mit folgenden Eigenschaften:
\begin{itemize}
	\item $V$ ist eine Menge von \alert{Knoten}.
	\item $E \subseteq V \times V$ ist eine Menge von \alert{Kanten}.
\end{itemize}
\end{definition}
\begin{block}<2->
{Intuition:}
\begin{itemize}
	\item Knoten sind zu ordnende \alert{Objekte} (Datensätze).
	\item Kanten sind \alert{Verweise} zwischen den Knoten (meist \alert{Pointer}).
	\item Unterscheidung: \alert{gerichtete} und \alert{ungerichtete} Graphen
	\begin{itemize}
		\item Bei ungerichteten Graphen haben Kanten keine Richtung.
		\item Zu jeder Kante gibt es eine Kante in die Rückrichtung.
	\end{itemize}
\end{itemize}
\end{block}
\end{frame}

\begin{frame}
\frametitle{\insertsection}
\begin{definition}[Baum]
Ein \alert{Baum} ist ein gerichteter Graph mit folgenden Eigenschaften:
\begin{itemize}
	\item Jedes Element hat höchstens einen Vorgänger.
	\item Es gibt genau ein Element ohne Vorgänger (die \alert{Wurzel}).
\end{itemize}
\begin{block}<2->
{Anders ausgedrückt:}
\glqq Ein Baum ist ein \alert{zusammenhängender gerichteter azyklischer Graph},
bei dem jedes Element höchstens einen Vorgänger hat.\grqq
\end{block}
\end{definition}
\begin{definition}[Binärbaum]<3->
Ein \alert{Binärbaum} ist ein Baum, bei dem jedes Element höchstens zwei Nachfolger hat.
\end{definition}
\end{frame}

\begin{frame}
\frametitle{\insertsection}
\begin{block}
{Sprechweise}
\begin{itemize}
	\item Nachfolger eines Knotens heißen \alert{Kinder}.
	\item Ein Knoten ohne Kinder ist ein \alert{Blatt}.
	\item Kinder werden meist in \alert{linke} und \alert{rechte} Kinder unterteilt.
	\item Ein Kind eines Knotes ist die Wurzel eines \alert{Teilbaums}.
\end{itemize}
\end{block}
\end{frame}