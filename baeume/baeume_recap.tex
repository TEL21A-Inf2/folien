\begin{frame}
\frametitle{\insertsection}
\begin{block}
{Zusammenfassung: Eigenschaften von Suchbäumen}
\begin{itemize}
	\item Datenstruktur zum effizienten Speichern von Listen.
	\item geordnete Speicherung von Werten:
		\begin{itemize}
			\item Linker Teilbaum enthält kleinere Werte als die Wurzel.
			\item Rechter Teilbaum enthält größere Werte als die Wurzel.
		\end{itemize}
	\item Neue Werte werden direkt an der richtigen Stelle eingefügt.
	\item Dadurch schnelles Suchen, Einfügen und Löschen von Werten.
\end{itemize}
\end{block}
\begin{block}<2->
{\alert{Problem:} Bäume können aus der Balance geraten.}
	\begin{itemize}
		\item Neue Elemente werden ggf.\ nur auf einer Seite angehängt.
		\item Der Baum wird zu einer einfach verketteten Liste.
		\item Man spricht von einem \alert{entarteten Baum}.
	\end{itemize}
\end{block}
	
\end{frame}

\begin{frame}
\frametitle{\insertsection}
%% Hier auch "Kinder", "Innere Knoten", "Wurzel" etc.?
%% - schon auf früherer Folie definiert
\begin{definition}[Tiefe eines Knotens in einem Baum]
	Die \alert{Tiefe} eines Knotens ist die Länge des Pfades bis zur Wurzel.
	\begin{itemize}
		\item Die Wurzel hat Tiefe 0.
		\item Die Kinder der Wurzel haben Tiefe 1.
		\item \ldots
	\end{itemize}
\end{definition}

\begin{definition}<2->[Höhe eines Baumes]
	Die \alert{Höhe} eines Baumes ist die maximale Länge eines Pfades von der Wurzel bis zu einem Blatt.
	\begin{itemize}
		\item alternativ: Die Höhe ist die maximale Tiefe eines Knotens.
	\end{itemize}
\end{definition}
\end{frame}

\begin{frame}
\frametitle{\insertsection}
\begin{definition}[Balancierter Baum]
	Ein Baum ist \alert{balanciert}, wenn für jeden Knoten gilt,
	dass sich die Höhe des linken und rechten Teilbaumes höchstens um ein bestimmtes Verhältnis unterscheiden.
	\begin{itemize}
		\item<2-> Dafür muss der Baum ggf.\ nach Einfügen oder Löschen eines Elements \alert{reorganisiert} werden.
		\item<3-> Hilfreiches Maß: \alert{Balancefaktor} $bf$ eines Knotens $k$.
		\begin{itemize}
			\item $bf(k)$ ist die Differenz zwischen der Höhe des rechten Teilbaumes und der Höhe des linken Teilbaumes.
			\item $bf(k) = h(\text{rechtes Kind}) - h(\text{linkes Kind})$
		\end{itemize}
	\end{itemize}	
\end{definition}
\end{frame}

\begin{frame}
	\begin{block}
	{Aufgabe: Entwerfen Sie Algorithmen, die \ldots}
	\begin{enumerate}
		\item \ldots die Tiefe eines Knotens in einem Baum bestimmen.
		\item \ldots die Höhe eines Baumes bestimmen.
		\item \ldots den Balancefaktor jedes Knotens ausgeben.
	\end{enumerate}
	\end{block}
\end{frame}