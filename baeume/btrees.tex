\begin{frame}
\frametitle{\insertsection}
\begin{block}
{Optimierung des Suchbaumprinzips: B-Bäume}
Bei einem \alert{B-Baum} kann ein Knoten mehr als zwei Kinder haben und mehr als einen Schlüssel tragen.
\begin{itemize}
	\item Trägt der Knoten $n$ Schlüssel, so hat er $n+1$ Kinder.
	\item Kind $0$ enthält Werte, die kleiner sind als der erste Schlüssel.
	\item Kind $1$ enthält Werte, die zwischen erstem und zweitem Schlüssel liegen usw.
\end{itemize}
\end{block}
\begin{block}<2->
{Eigenschaften}
\begin{itemize}
%	\item Die Schlüssel innerhalb der Knoten müssen sortiert sein.
	\item Die Anzahl der Schlüssel pro Knoten ist variabel
		\begin{itemize}
			\item Meist zwischen $n$ und $2n$ für vorgegebene Zahl $n$.
		\end{itemize}
	\item Alle Blätter haben die gleiche Tiefe.
	\begin{itemize}
		\item ggf. Zusatzschlüssel in inneren Knoten benutzen.
	\end{itemize}
\end{itemize}
\end{block}
\begin{block}<3->
{B-Bäume sind eine typische Datenstruktur in Datenbanken und Dateisystemen.}
\end{block}
\end{frame}