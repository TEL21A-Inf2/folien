\begin{frame}
\frametitle{\insertsection}
\begin{block}
{Bisheriger Ansatz: Bäume als Listen}
\begin{itemize}
	\item[\positive] Suchbaumeigenschaft garantiert korrekte Sortierung.
	\item[\positive] Balancierungsoperationen für schnellen Zugriff.
	\item[\negative] Problem: Selbst AVL-Bäume können noch unnötig hoch werden.
\end{itemize}
\end{block}
\begin{block}<2->
{Alternatives Gütekriterium: Vollständigkeit}
\begin{itemize}
	\item Versuche, den Baum möglichst perfekt zu balancieren.
	\item Verzichte dafür auf korrekte Sortierung.
	\begin{itemize}
		\item Baum sollte immer noch partiell sortiert sein.
	\end{itemize}
\end{itemize}
\end{block}
\end{frame}

\begin{frame}
\frametitle{\insertsection}
\begin{definition}[vollständiger Binärbaum]
	Ein vollständiger Binärbaum ist ein Binärbaum, bei dem alle Ebenen voll
	besetzt sind.
	\begin{itemize}
		\item Ausnahme: Die unterste Ebene muss nicht vollständig sein.
			In diesem Fall sind die Knoten von links durchgehend besetzt.
	\end{itemize}
\end{definition}
\begin{block}<2->
{Intuition}
\begin{itemize}
	\item Jeder Knoten (außer den Blättern) hat zwei Kinder.
	\item Vollständige Bäume sind der Idealfall:\\
		Perfekt balanciert und minimale Suchtiefe.
\end{itemize}
\end{block}
\end{frame}


\begin{frame}
\frametitle{\insertsection}

\begin{definition}[Heap]
	Ein \alert{Heap} ist ein vollständiger Binärbaum, 
	bei dem der Wert jedes Knotens kleiner ist als der seiner Kinder.
\end{definition}
\begin{block}
{Beobachtungen}
\begin{itemize}
	\item Die Wurzel ist das kleinste Element.
	\item Der Baum ist \alert{partiell sortiert}:\\
		Beim Absteigen werden die Elemente größer.
\end{itemize}
\end{block}
\begin{block}<2->
{Alternative Definitionen:}
\begin{itemize}
	\item Ein Heap wie oben definiert heißt \alert{min-Heap}.
	\item \alert{Max-Heap}: Die Wurzel ist \alert{größer} als ihre Kinder.
\end{itemize}
\end{block}
\end{frame}


\endinput

\begin{frame}
\frametitle{\insertsection}
\begin{block}
{}
\end{block}
\end{frame}
