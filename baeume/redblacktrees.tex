\begin{frame}
\frametitle{\insertsection}
\begin{block}
{Weitere Idee: Abschwächung des AVL-Prinzips}
\begin{itemize}
	\item Keine perfekte, sondern näherungsweise Balancierung.
	\item<2-> Idee: Knoten in rot oder schwarz einfärben.
	\begin{itemize}
		\item Baum ist balanciert, wenn man nur schwarze Knoten betrachtet.
		\item Anzahl der roten Knoten ist begrenzt.
	\end{itemize}
	\item<3-> Vorteil: Es muss nicht jedes Mal neu balanciert werden.
\end{itemize}
\end{block}
\begin{definition}<4->[Rot-Schwarz-Bäume]
Ein \alert{Rot-Schwarz-Baum} ist ein Binärbaum, bei dem jeder Knoten eine Farbe (Rot oder Schwarz) hat.
\begin{itemize}
	\item Jedes Blatt ist schwarz.
	\item Ein roter Knoten hat nur schwarze Kinder.
	\item Jeder Pfad von einem Knoten zu seinen Blättern hat die gleiche Anzahl schwarzer Knoten.
\end{itemize}
\end{definition}
\end{frame}