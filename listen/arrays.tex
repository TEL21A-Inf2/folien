\begin{frame}
\frametitle{\insertsection}
\begin{block}<1->
{Array: Zusammenhängender Bereich im Speicher}
\end{block}
\vspace{-1em}
\begin{block}<2->
{Vorteile}
\begin{itemize}
	\item Zugriffe i.d.R.\ sehr schnell
	\item wahlfreier Zugriff möglich
	\item kann leicht durchlaufen werden (Pointerberechnungen)
\end{itemize}
\end{block}
\begin{block}<3->
{Nachteile}
\begin{itemize}
	\item feste Größe
	\item bei Verletzung der Grenzen können Fehler auftreten
	\item Größenänderung unmöglich bzw.\ teuer
	\item evtl.\ schlechte Ausnutzung des Speichers
\end{itemize}
\end{block}
\end{frame}

\begin{frame}
\frametitle{\insertsection}
\begin{block}<1->
{Lösungsansatz: Dynamische Arrays}
\end{block}
\vspace{-1em}
\begin{block}<2->
{Idee: Zugriffe nicht direkt mittels \code{\lbrack\rbrack}, sondern durch Funktionen:}
\begin{itemize}
	\item<3-> z.B.: \code{array.get(i)} und \code{array.set(i,el)}
	\begin{itemize}
		\item \code{array.get(i)} liest im \code{array} an Stelle \code{i}
		\item prüft dabei, ob \code{i} ein gültiger Index ist
		\item \code{array.set(i,el)} schreibt Element \code{el} an Stelle \code{i}
		\item verändert ggf.\ die Größe des Arrays
	\end{itemize}
	\item<4-> z.B.: \code{add}, \code{remove}
	\begin{itemize}
		\item \code{add} fügt ein Element am Ende der Liste hinzu.
		\item \code{remove} löscht das letzte Element der Liste.
		\item \alert{Standardoperationen auf Listen}
	\end{itemize}
\end{itemize}
\end{block}
\end{frame}

\begin{frame}
\frametitle{\insertsection}
\begin{block}
{Implementierung dynamischer Arrays}
\end{block}
\vspace{-1em}
\begin{block}
{Record-Datentyp (Klasse oder Struct), der ein Array enthält}
\begin{itemize}
	\item Felder:
	\begin{itemize}
		\item Pointer/Referenz auf das Array
		\item Größe des reservierten Speichers
		\item Tatsächliche Anzahl der Elemente
	\end{itemize}
	\item<2-> Member-Funktionen für Zugriff:
	\begin{itemize}
		\item \code{add}
		\item \code{remove}
		\item \code{get}
		\item \code{reallocate} (intern von \code{add} benutzt)
	\end{itemize}
\end{itemize}
\end{block}
\end{frame}