\begin{frame}
\frametitle{\insertsection}
\begin{block}
{Daten sind abstrakter Bestandteil in Listen}
\begin{itemize}
	\item in theoretischen Beispielen meist \code{int}
	\item in der Praxis oft komplexer
	\begin{itemize}
	    \item zusammengesetzte Daten (Record-Datentypen)
	    \item Schlüssel können z.B. auch Strings o.Ä. sein.
	\end{itemize}
\end{itemize}
\end{block}

\begin{block}<2->
{Objektorientierter Ansatz für einheitlichen Entwurf:}
\begin{itemize}
    \item Daten werden immer in einem Datentyp \code{element} \alert{gekapselt}.
    \item Elemente sind Bestandteile der Listen, Arrays etc.
    \item Zugriff auf die Daten mittels Funktionen.
   \item \alert{Die Implementierung der eigentlichen Liste, des Arrays etc.\
        muss dabei nicht verändert werden.}
   \begin{itemize}
     \item In Java: Meist werden Generics verwendet.
   \end{itemize}
\end{itemize}
\end{block}
\end{frame}

\endinput

\begin{frame}
\frametitle{\insertsection}
\begin{block}
{}
\end{block}
\end{frame}