\begin{frame}
\frametitle{\insertsection}
\begin{block}
{Doppelte Verkettung: Pointer auf Nachfolger und Vorgänger}
\begin{itemize}
	\item<2-> Idee: Elemente der Liste bestehen aus drei Teilen
	\begin{itemize}
		\item Daten
		\item Pointer auf das nächste Element
		\item Pointer auf das vorhergehende Element
	\end{itemize}
\end{itemize}
\end{block}
\begin{block}<3->
{Vor- und Nachteile:}
\begin{itemize}
	\item effizienter bei wiederholten Zugriffen auf benachbarte Elemente
	\item etwas mehr Speicherverbrauch als einfach verkettete Liste
	\begin{itemize}
		\item kann i.d.R.\ vernachlässigt werden
	\end{itemize}
\end{itemize}
\end{block}
\begin{block}<4->
{Verwaltung durch Dummy-Element}
\begin{itemize}
	\item Nutze ein spezielles leeres Element (\code{HEAD})
	\item markiert Anfang und Ende der Liste
\end{itemize}
\end{block}
\end{frame}

\begin{frame}
\frametitle{\insertsection}
\begin{block}
{Implementierung doppelt verketteter Listen}
\end{block}
\vspace{-1em}
\begin{block}
{Elemente: \code{struct} mit Daten und Pointer auf Nachbarn}
\end{block}
\vspace{-1em}
\begin{block}
{Liste: \code{struct} mit Pointer auf den Kopf der Liste}
\begin{itemize}
	\item Felder:
	\begin{itemize}
		\item Pointer auf das \code{HEAD}-Element
	\end{itemize}
	\item<2-> Basisfunktionen für Zugriff:
	\begin{itemize}
		\item \code{add}
		\item \code{remove}
		\item \code{get}
	\end{itemize}
\end{itemize}
\end{block}
\end{frame}