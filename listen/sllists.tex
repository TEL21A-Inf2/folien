\begin{frame}
\frametitle{\insertsection}
\begin{block}
{Verkettete Liste: Elemente stehen verteilt im Speicher}
\begin{itemize}
	\item<2-> Idee: Elemente der Liste bestehen aus zwei Teilen
	\begin{itemize}
		\item Daten
		\item Pointer auf das nächste Element
	\end{itemize}
\end{itemize}
\end{block}
\begin{block}<3->
{Vorteile:}
\begin{itemize}
	\item Größe ist nicht fest vorgegeben
	\item sehr effiziente Speicherausnutzung
	\item Einfügen und Löschen von Elementen sehr effizient
\end{itemize}
\end{block}
\begin{block}<4->
{Nachteile:}
\begin{itemize}
	\item Kein wahlfreier Zugriff
	\item Durchlauf ineffizient
\end{itemize}
\end{block}
\end{frame}


\begin{frame}
\frametitle{\insertsection}
\begin{block}
{Implementierung einfach verketteter Listen}
\end{block}
\vspace{-1em}
\begin{block}
{Elemente: \code{struct} mit Daten und Pointer auf Nachfolger}
\end{block}
\vspace{-1em}
\begin{block}
{Liste: \code{struct} mit Pointer auf den Anfang der Liste}
\begin{itemize}
	\item Felder:
	\begin{itemize}
		\item Pointer auf das erste Element der Liste
		\item evtl.\ weitere Pointer (für bessere Performance)
	\end{itemize}
	\item<2-> Basisfunktionen für Zugriff:
	\begin{itemize}
		\item \code{add}
		\item \code{remove}
		\item \code{get}
	\end{itemize}
	\item<3-> Ende wird durch ein Dummy-Element markiert.
	\begin{itemize}
	    \item Sentinel-Prinzip, vgl.\ terminierende Null bei Strings
	\end{itemize}
\end{itemize}
\end{block}
\end{frame}