\begin{frame}
\frametitle{\insertsection}
\begin{block}
{Stacks: \emph{Stapel-} oder \emph{Kellerspeicher}}
\begin{itemize}
	\item Elemente werden \emph{gestapelt}.
	\item Nur das zuletzt eingefügte Element ist zugänglich
	\begin{itemize}
		\item \glqq Last-In-First-Out\grqq\ (LIFO)
	\end{itemize}
	\item<2-> Standardoperationen: \code{push}, \code{pop} und \code{top}
	\begin{itemize}
		\item \code{push} fügt ein Element hinzu
		\item \code{pop} entfernt das oberste Element
		\item \code{top} liefert das oberste Element zurück
	\end{itemize}
	\item<3-> Implementierung durch Arrays oder verkettete Listen
	\item<4-> Anwendungsbeispiele:
	\begin{itemize}
		\item Der \emph{Stack} im Hauptspeicher
		\item Pufferspeicher bei rekursiven Problemen (z.B. Damenproblem)
		\item Einfaches Speichermodell in Kleinstrechnern (z.B. Taschenrechner)
	\end{itemize}
\end{itemize}
\end{block}
\end{frame}