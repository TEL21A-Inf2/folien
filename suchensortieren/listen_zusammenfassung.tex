\begin{frame}
\frametitle{ \insertsection}
\begin{block}
{Listen-Datentypen}
\begin{itemize}
    \item allgemeine Form des Arrays
    \item Zugriff auf Elemente über Index
    \item elementare Funktionen wie \code{push\_back} etc.
    \item Implementierung unabhängig von konkreten Daten
        \begin{itemize}
            \item deshalb meist nur \code{int} als Beispiel
        \end{itemize}
    \item Konkreter Listentyp ist nur von technischer Bedeutung:
    \begin{itemize}
        \item Unterschiede bei Performance, Speichernutzung etc.
        \item im Idealfall keine Unterschiede bei Benutzung
    \end{itemize}
\end{itemize}
\end{block}
\end{frame}

\begin{frame}
\frametitle{\insertsection}
\begin{block}
{Meist vielfältige Datensätze}
\begin{itemize}
    \item z.B.\ Adressbuch, Sensordaten, Figuren in Spielen ...
\end{itemize}
\end{block}
\begin{block}<2->
{Eigenschaften von Datenelementen}
\begin{itemize}
    \item \alert{Daten müssen vergleichbar sein}
    \item \alert{Daten haben eine Ordnung!}
    \begin{itemize}
        \item einzelne Schlüssel oder gesamter Datensatz
        \item Ordnung/Schlüssel vereinfachen Suchen und Sortieren.
    \end{itemize}
\end{itemize}
\end{block}

\end{frame}

\endinput

\begin{frame}
\frametitle{\insertsection}
\begin{block}
{}
\end{block}
\end{frame}