\begin{frame}
\frametitle{\insertsection}
\begin{block}
{Quicksort}
\begin{itemize}
	\item Idee: Sortiere die Liste in \alert{große Zahlen} und \alert{kleine Zahlen} vor.
	\item Sortiere die Teillisten und hänge sie wieder zusammen.
\end{itemize}
\end{block}
\begin{block}<2->
{Algorithmus:}
\begin{enumerate}
	\item Wähle ein \alert{Pivotelement}.
	\item Erzeuge zwei (möglichst gleich große) Teillisten:
	\begin{itemize}
		\item Elemente die \alert{kleiner} sind als das Pivotelement.
		\item Elemente die \alert{größer} sind als das Pivotelement.
	\end{itemize}
	\item Sortiere rekursiv die beiden Teillisten.
\end{enumerate}
\end{block}
\begin{block}<3->
{Wie wählt man das Pivotelement?}
\begin{itemize}
	\item Mitte des Bereichs, mittleres Element oder erstes Element?
	\item Hängt von Struktur der Daten ab.
\end{itemize}
\end{block}
\end{frame}

\begin{frame}
\frametitle{\insertsection}
\begin{block}
{Mergesort}
\begin{itemize}
	\item Lösung ohne das Pivot-Problem: \\ Ohne Vorsortierung direkt die Liste zerlegen.
	\item Erst beim Zusammensetzen die Liste sortieren.
\end{itemize}
\end{block}
\begin{block}
{Algorithmus}<2->
\begin{enumerate}
	\item Zerlege die Liste in gleich große Teillisten.
	\item Sortiere die Teillisten rekursiv.
	\item \alert{Mische} die Teillisten zusammen.
\end{enumerate}
\end{block}
\end{frame}

\endinput

\begin{frame}
\frametitle{\insertsection}
\begin{block}
{}
\end{block}
\end{frame}