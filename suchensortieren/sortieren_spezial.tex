\begin{frame}
\frametitle{\insertsection}
\begin{block}
{Shellsort}
\begin{itemize}
	\item Problem bei Insertionsort: Elemente müssen zum Einsortieren oft über weite Strecken verschoben werden.
	\item Andererseits: Insertionsort arbeitet sehr effizient, wenn die Liste schon fast sortiert ist.
\end{itemize}
\end{block}
\begin{block}<2->
{Idee: Vorab eine grobe Ordnung herstellen.}
\begin{itemize}
	\item Betrachte zunächst nur jedes $i$-te Element.
	\item Verringere $i$ in jedem Schritt bis $i=1$.
\end{itemize}
\end{block}
\end{frame}

\begin{frame}
\frametitle{\insertsection}
\begin{block}
{Bucketsort}
\begin{itemize}
	\item Vor- oder Teilsortierung mittels Kategorien
	\item Oft ist der Datenbereich strukturiert, Elemente können kategorisiert werden.
	\item Vollständige Sortierung evtl.\ nicht notwendig.
\end{itemize}
\end{block}
\begin{block}<2->
{Algorithmus:}
\begin{enumerate}
	\item Sortiere Daten nach Kategorien in sog.\ \alert{Buckets}
	\begin{itemize}
		\item Kategorien können Jahreszahlen, Anfangsbuchstaben, Zahlenbereiche etc.\ sein.
	\end{itemize}
	\item Sortiere jeden Bucket einzeln mit einem klassischen Verfahren.
\end{enumerate}
\end{block}
\end{frame}

\endinput

\begin{frame}
\frametitle{\insertsection}
\begin{block}
{}
\end{block}
\end{frame}