\begin{frame}
\frametitle{\insertsection}
\begin{block}
{Ziel: Herausfinden, ob und an welcher Stelle in einer Liste ein bestimmtes Element steht.}
\end{block}
\vspace{-1em}
\begin{block}<2->
{Suchen in unsortierten Listen:}
\begin{itemize}
	\item Einzige Möglichkeit: Die Liste linear durchsuchen.
\end{itemize}
\end{block}
\begin{block}<3->
{Suche in sortierten Listen: \alert{Binäre Suche}}
\begin{enumerate}
	\item Vergleich mit mittlerem Element.
	\item Ist das gesuchte Element kleiner, suche in linkem Teil weiter.
	\item Ist das gesuchte Element größer, suche in rechtem Teil weiter.
\end{enumerate}
\end{block}
\begin{block}<4->
{Die binäre Suche kann effizient \alert{rekursiv} implementiert werden.}
\end{block}
\end{frame}

\endinput

\begin{frame}
\frametitle{\insertsection}
\begin{block}
{}
\end{block}
\end{frame}